\documentclass{beamer} %[handout]
\usetheme{CambridgeUS}
\usecolortheme{whale}
\usepackage{color}
\usepackage{xcolor}
\usepackage[makeroom]{cancel}
\usepackage{graphicx} 
\setbeamercolor{frametitle}{fg=black}
\usepackage{mathtools}
\usepackage{amssymb}
\usepackage{amsmath}
\usepackage{amsthm}
\usepackage{graphicx}
\usepackage{fancyvrb}
\usepackage{listings}
\usepackage{tikz}
\usepackage[utf8]{inputenc}
\usepackage[slovene]{babel}
\usepackage{bm}

\usetikzlibrary{shapes}

\usetikzlibrary{decorations.text,decorations.pathreplacing}

\lstset{frame=tb,
  language=R,
  aboveskip=3mm,
  belowskip=3mm,
  showstringspaces=false,
  columns=flexible,
  basicstyle={\tiny\ttfamily},
  numbers=none,
  numberstyle=\tiny\color{gray},
%  keywordstyle=\color{blue},
%  identifierstyle=\color{yellow},
  breaklines=true,
    literate={->}{$\rightarrow$}{2}
           {°}{$\epsilon$}{1},
  breakatwhitespace=true
  tabsize=3
}

\usetikzlibrary{arrows,positioning} 
\tikzset{
    %Define standard arrow tip
    >=stealth',
    %Define style for boxes
    punkt/.style={
           rectangle,
           rounded corners,
           draw=black, very thick,
           text width=6.5em,
           minimum height=2em,
           text centered},
    % Define arrow style
    pil/.style={
           ->,
           thick,
           shorten <=2pt,
           shorten >=2pt,}
}

\makeatletter
\def\th@mystyle{%
    \normalfont % body font
    \setbeamercolor{block title example}{bg=orange,fg=white}
    \setbeamercolor{block body example}{bg=orange!20,fg=black}
    \def\inserttheoremblockenv{exampleblock}
  }
\makeatother
\theoremstyle{mystyle}
\newtheorem*{remark}{Primer}

\makeatletter
\def\th@mystylet{%
    \normalfont % body font
    \setbeamercolor{block title example}{bg=purple,fg=white}
    \setbeamercolor{block body example}{bg=purple!20,fg=black}
    \def\inserttheoremblockenv{exampleblock}
  }
\makeatother
\theoremstyle{mystylet}
\newtheorem*{analysis}{Primer}

\defbeamertemplate{description item}{align left}{\insertdescriptionitem\hfill}

\date{2018}
\author[Bayesian statistics]{\v{Z}iga Sajovic \\
\href{mailto:ziga.sajovic@xlab.si}{ziga.sajovic@xlab.si}}
\institute{\href{https://www.xlab.si}{XLAB d.o.o.}}

\linespread{1.2}
\renewcommand{\thefootnote}{\roman{footnote}}
\newcommand{\K}{\mathcal{K}}
\newcommand{\D}{\mathcal{D}}
\newcommand{\C}{\mathcal{C}}
\newcommand{\B}{\mathcal{B}}
\newcommand{\pr}[1]{P\{#1\}}
\addto\captionsslovene{\renewcommand{\proofname}{Proof}}
%\setbeameroption{show notes}

\title[Algebra of Uncertainty]{Algebra of Uncertainty}

\begin{document}

% Add why subjective probability, as to discern on the ability to quantify uncertainty in a an opinion (question from X on first lectures)

\begin{frame}
%Bayesian Statistics
\titlepage
\end{frame}

\begin{frame}{Quantifying the degrees of belief}
\begin{minipage}[]{\linewidth}
We can quantify our beliefs about an uncertain event $A$ by being willing to trade them for a possibility of something quantifiable.
\end{minipage}
\vfill
\begin{minipage}[]{\linewidth}
\begin{minipage}[]{0.45\linewidth}
{\textbf{The buy}}
We are willing to trade certainly giving out the quantity $\pr{A}$ in exchange for a \emph{ticket} that's redeemable for \emph{one unit} of the quantity, in case $A$ holds true.

\end{minipage}
\hfill
\begin{minipage}[]{0.45\linewidth}
{\textbf{The sell}}
We are willing to trade a \emph{ticket} that's redeemable for \emph{one unit} of the quantity in case $A$ holds true, in exchange for certainly receiving the quantity $\pr{A}$.
\end{minipage}
\end{minipage}
\vfill
\begin{minipage}[]{\linewidth}
The quantity being traded may be anything that's quantifiable to us. Note that we sell and buy tickets on $A$ at the same price $\pr{A}$.
\end{minipage}
\end{frame}

\begin{frame}{The Dutch Book ($\D$)}
\begin{minipage}[]{\linewidth}
The term \emph{dutch book} ($\D$) labels a quantification of our beliefs about events $A_i$, by which any possible trade based on such beliefs will result in our \emph{loss}, regardless of the truth attached to the final outcome.
\end{minipage}
\vfill
\begin{minipage}[]{\linewidth}
\begin{minipage}[]{0.45\linewidth}
With quantifiable subjective beliefs, the term is widely applicable.\bigskip

{\textbf{Ex.:}} in econometrics $\D$ is a trade leaving one party strictly worse off.
\end{minipage}
\hfill
\begin{minipage}[]{0.45\linewidth}
{\textbf{Ex.:}} in courtship $\D$ could be a set of beliefs about the opposite sex, that leads to a sequence of interactions that always leaves you rejected and alone.% in the friend-zone.
\end{minipage}
\end{minipage}
\end{frame}

\begin{frame}{Coherence for disjoint events}
\begin{minipage}[]{\linewidth}
\begin{minipage}[]{0.55\linewidth}
{\textbf{Elements of discussion}}
%\setbeamertemplate{description item}[align left]
\begin{itemize}\itemsep0.8em
\item[$A_i$]
is the set of disjoint events $A_i$
%$$\forall_{i\ne j}(A_i\cap A_j=\emptyset)$$
\item[$P$] are the prices $\pr{A_i}$ of the ticket on $A_i$
\item[$\C$] is the set of rules $R_i$
\end{itemize}
\end{minipage}
\hfill
\begin{minipage}[]{0.3\linewidth}
\centering
$$
\C
=
% \begin{pmatrix}
% R_1,\\
% R_2,\\
% \vdots\\
% R_n
% \end{pmatrix}
\left\{\!\begin{matrix}
R_1, \\[1ex]
R_2, \\[1ex]
\vdots \\[1ex]
R_n
\end{matrix}\right\}
\quad
$$
\end{minipage}
\end{minipage}
\vfill
\begin{minipage}[]{\linewidth}
We seek a coherent set of rules $\C$ by which to prescribe prices $\pr{A_i}$ to events $A_i$ in a way that prevents the possibility of being in a dutch book $\D$,
$$\C\iff\neg\D.$$
\end{minipage}
\end{frame}

\begin{frame}{Constructing the argument}
\begin{minipage}[]{0.3\linewidth}
$$
\C\iff\neg \D
\sim
\begin{cases}
\C\impliedby\neg\D \\
\C\implies\neg\D
\end{cases}
$$
\end{minipage}
\hfill
\begin{minipage}[]{0.5\linewidth}
\begin{itemize}\itemsep1.2em
\item[$\impliedby$] If it is impossible for us to be in a dutch book ($\D$), then we must have been coherent ($\C$).


\item[$\implies$] If we are coherent ($\C$), then we cannot end up in a dutch book ($\D$).
\end{itemize}
\end{minipage}
\end{frame}

\begin{frame}{$\C\impliedby\neg\D$}
This statement is easier to frame as a story, so we being with it, to build our intuition and the set of rules itself.

$$\C\impliedby\neg\D$$
$$\iff$$
$$\neg\C\implies\D$$

By way of contrapositive, we can prove that by not being coherent ($\C$) we can end up in a dutch book ($\D$).
\end{frame}

\begin{frame}{$\neg\C\implies\D;\quad$ non-negativity and certain ($A_{C}$) events}


\begin{minipage}[]{\linewidth}
\begin{minipage}[]{0.45\linewidth}
{\textbf{Any event$\implies \pr{A}\ge0$}}
 \\
Imagine \emph{You} sell me a ticket for $A$ at price $-p<0$. No mater what happens, your loss is at least $p$.

Thus
$$\pr{A}<0\implies\D$$
\end{minipage}
\hfill
\begin{minipage}[]{0.45\linewidth}
{\centering\textbf{A is certain$\implies\pr{A}=1$}}
 \\
Imagine \emph{You} sell me a ticket for $A$ at price $p<1$. No mater what happens, your loss is at least $1-p$.

Thus

$$\pr{A_{C}}\ne1\implies\D$$

\end{minipage}

% \vfill

% \begin{minipage}[]{\linewidth}
% A direct consequence of 
% %\centering
% % {\textbf{Any event$\implies \pr{A}\ge0$}}

% % Imagine \emph{You} sell me a ticket for $A$ at price $-p<0$. Your loss is at least $p$.
% % \vspace{-5px}$$\pr{A}<0\implies\D$$
% \end{minipage}
\end{minipage}

\end{frame}

\begin{frame}{$\neg\C\implies\D;\quad$ finite additivity ($FA_>$)}

% $A_1\cap A_2=\emptyset\implies\pr{A_1\cup A_2}=\pr{A_1}+\pr{A_2}$

\begin{minipage}[]{0.45\linewidth}
\begin{minipage}[]{\linewidth}
%\centerline{$\bm{\neg FA_>}$}
{$\bm{\underbrace{\pr{A_1\cup A_2}>\pr{A_1}+\pr{A_2}}_{\neg FA_>}}$}
\bigskip

Imagine \emph{You} buy from me a ticket for $A_1\cup A_2$ at a price $p$ and sell me tickets for both $A_1$ and $A_2$ at a price $p_1+p_2$. 
\end{minipage}
\vfill
\bigskip
\begin{minipage}[]{\linewidth}
Thus
$$\neg FA_>\implies \D$$
\end{minipage}
\end{minipage}
\hfill
\begin{minipage}[]{0.45\linewidth}
\begin{itemize}\itemsep1.2em
\item If neither event happens there is no payout and the net loss is the price of the tickets
$$-p+p_1+p_2<0$$

\item If either $A_i$ happens we both pay each other $1$, with net loss being the same
$$-p+p_1+p_2+1-1<0$$

\end{itemize}
\end{minipage}

\end{frame}

\begin{frame}{$\neg\C\implies\D;\quad$ finite additivity ($FA_<$)}

% $A_1\cap A_2=\emptyset\implies\pr{A_1\cup A_2}=\pr{A_1}+\pr{A_2}$

\begin{minipage}[]{0.45\linewidth}
\begin{minipage}[]{\linewidth}
%\centerline{$\bm{\neg FA_<}$}
{$\bm{\underbrace{\pr{A_1\cup A_2}<\pr{A_1}+\pr{A_2}}_{\neg FA_<}}$}
\bigskip

Imagine \emph{You} sell me a ticket for $A_1\cup A_2$ at price a $p$ and buy from me tickets for both $A_1$ and $A_2$ at a price $p_1+p_2$. 
\end{minipage}
\vfill
\bigskip
\begin{minipage}[]{\linewidth}
Thus
$$\neg FA_<\implies \D$$
\end{minipage}
\end{minipage}
\hfill
\begin{minipage}[]{0.45\linewidth}
\begin{itemize}\itemsep1.2em
\item If neither event happens there is no payout and the net loss is the price of the tickets
$$-p_1-p_2+p<0$$

\item If either $A_i$ happens we both pay each other $1$, with net loss being the same
$$-p_1-p_2+p+1-1<0$$

\end{itemize}
\end{minipage}

\end{frame}

\begin{frame}{$\neg\C\implies\D;\quad$ finite additivity ($FA$)}
\begin{minipage}{\linewidth}
\begingroup
\setlength{\belowdisplayskip}{2pt}
{$$\bm{\underbrace{\pr{A_1\cup A_2}=\pr{A_1}+\pr{A_2}}_{FA}}$$}
\endgroup
\end{minipage}
\vfill
\begin{minipage}[]{0.45\linewidth}
\begingroup
\setlength{\abovedisplayskip}{3pt}
\setlength{\belowdisplayskip}{3pt}
% \centerline{$\bm{FA}$}
% {$\bm{\pr{A_1\cup A_2}=\pr{A_1}+\pr{A_2}}$}

Combining the two arguments,
$$\neg FA_<\land \neg FA_<\iff \neg FA$$
we arrive at the conclusion
$$\pr{A_1\cup A_2}\ne\pr{A_1}+\pr{A_2}$$
$$\implies\D$$
stating that a lack of finite additivity, leads to a dutch book.
\endgroup
\end{minipage}
\hfill
\begin{minipage}[]{0.45\linewidth}
By induction, the argument is easily extended to any finite number of disjoint events
$$\pr{\cup_{i=0}^N A_i}\ne\sum_{i=0}^N \pr{A_i}\implies\D$$
Thus
$$\neg FA\implies \D$$
\end{minipage}

\end{frame}

\begin{frame}{$\C\impliedby\neg\D$ proven by contrapositive}
\begin{minipage}[]{\linewidth}
\begin{minipage}[]{0.45\linewidth}
$$\pr{A}<0\implies\D$$
$$\pr{A_{C}}\ne1\implies\D$$
$$\neg FA\implies \D$$
\end{minipage}
\hfill
\begin{minipage}{0.05\linewidth}
\centering
$$\iff$$
\end{minipage}
\hfill
\begin{minipage}[]{0.45\linewidth}
$$\pr{A}\ge0\impliedby\neg\D$$
$$\pr{A_{C}}=1\impliedby\neg\D$$
$$FA\impliedby \neg\D$$
\end{minipage}
\end{minipage}
\vfill
\begin{minipage}{\linewidth}
Collecting the rules into a set
\vspace{-7px}
$$\C=\{\pr{A}\ge0,\pr{A_{C}}=1, FA\},$$

\vspace{-7px}we state that if it is impossible to end up in a dutch book then the set of rules $\C$ must have been obeyed when making the trade,
\vspace{-5px}
$$\C\impliedby\neg\D.$$
\end{minipage}
\end{frame}

\begin{frame}{$\C\implies\neg\D;\quad$ Describing the trades}
\begin{minipage}{\linewidth}
\begin{minipage}[]{0.45\linewidth}
{\centering\textbf{Tickets bought for $p_i$}}
\begin{itemize}\itemsep0.5em
\item[$A_i$] If $A_i$ occurred your payout for $\alpha_i$ tickets is $\alpha_i(1-p_i)$.
\item[$\neg A_i$] If $\neg A_i$ occurred your payout for $\alpha_i$ tickets is $\alpha_i(-p_i)$.
\end{itemize}
\end{minipage}
\hfill
\begin{minipage}[]{0.45\linewidth}
{\centering\textbf{Tickets sold for $p_i$}}
\begin{itemize}\itemsep0.5em
\item[$A_i$] If $A_i$ occurred your payout for $\beta_i$ tickets is $-\beta_i(1-p_i)$.
\item[$\neg A_i$] If $\neg A_i$ occurred your payout for $\beta_i$ tickets is $\beta_ip_i$.
\end{itemize}
\end{minipage}
\end{minipage}
\vfill
\begin{minipage}{\linewidth}
After allowing trades on all events, our winnings are described by
\vspace{-5px}
$$W=\sum_{i=1}^N\lambda_i(I_{A_i}-p_i)$$

\vspace{-5px}where $\lambda_i=\alpha_i-\beta_i$ (or $0$ if no trades on $A_i$) and $I_{A_i}$ is the indicator of $A_i$.
\end{minipage}
\end{frame}

\begin{frame}{$\C\implies\neg\D;\quad$ The expected payout}
\begin{minipage}{\linewidth}
\begin{minipage}[]{0.45\linewidth}
Noting that the expectation of an indicator of $A_i$
$$E(I_{A_i})=\sum_{k=1}^Np_kI_{A_i}(A_k)=p_i$$
is its probability, we see that the expected payout is $E(W)=0$.
\end{minipage}
\hfill
\begin{minipage}[]{0.45\linewidth}
\begin{equation*}
\begin{split}
E(W) & =E(\sum_{i=1}^N\lambda_i(I_{A_i}-p_i))\\
& =\sum_{i=1}^NE(\lambda_i(I_{A_i}-p_i)) \\
& =\sum_{i=1}^N\lambda_iE(I_{A_i}-p_i) \\
& = 0
\end{split}
\end{equation*}
\end{minipage}
\end{minipage}
\end{frame}

\begin{frame}{$\C\implies\neg\D;\quad$ A supportive theorem on expectations}

\begin{theorem}
If $X$ is a non-trivial random variable, then
$$\min{X}=x_{min}<E(X)<\max{X}=x_{max}.$$
\end{theorem}
\begin{proof}
$$\min{X}=x_{min}=\sum_{i=0}^np_ix_{min}<E(X)<\sum_{i=0}^np_ix_{max}=x_{max}=\max{X}$$
\end{proof}
\end{frame}

\begin{frame}{$\C\implies\neg\D;\quad$ An immediate corollary}

\begin{corollary}[]
If X is non-trivial, there is some positive probability $\epsilon_1 > 0$ that $X$
exceeds its expectation $E(X)$ by a fixed amount $\eta_1 > 0$, and positive probability $\epsilon_2 > 0$ that
$E(X)$ exceeds $X$ by a fixed amount $\eta_2 > 0$.
\end{corollary}
\begin{proof}
Denote by $p_1$ the probability of $x_{min}$ and $p_2$ the probability of $x_{max}$. Than $\eta_1=x_{max}-E(X)>0$ and $\eta_2=E(X)-x_{min}>0$, with $\epsilon_1=p_1$ and $\epsilon_2=p_2$.
\end{proof}
\end{frame}

\begin{frame}{$\C\implies\neg\D;\quad$ Implications of $E(W)$}
\begin{minipage}{\linewidth}
\begin{minipage}[]{0.45\linewidth}
{\textbf{Implications of the $E(W)=0$}}
\begin{itemize}\itemsep0.5em
\item Either $W$ is trivial and there is no uncertainty, no gambles and no loss, or
\item there is a positive probability $\epsilon$, that you will gain at least the amount $\eta$, as by Corollary, i.e. no dutch book.
\end{itemize}
\end{minipage}
\hfill
\begin{minipage}[]{0.45\linewidth}
$$\bm{E(W)=0\implies \neg\D}$$
The calculation of the expected payout was dependent on the rules of probability which coincide with the rules of coherence $\C$. Thus

$$\C\implies\neg\D,$$

proving the statement.
\end{minipage}
\end{minipage}
\end{frame}

\begin{frame}{$\C\iff\neg\D;\quad$ Theorem of Coherence}
\begin{minipage}[]{0.5\linewidth}
{\textbf{Theorem} (Coherence)}
Your prices $\pr{A_i}$ at which you are willing to buy and sell tickets cannot lead you into a dutch book if and only if they are coherent,

$$\C\iff\neg\D.$$
\end{minipage}
\hfill
\begin{minipage}[]{0.3\linewidth}
\centering
$$
\C
=
\left\{\!\begin{matrix}
\pr{A_i}\ge0, \\[1ex]
\pr{A_{C}}=1, \\[1ex]
FA
\end{matrix}\right\}
\quad
$$
\end{minipage}
\end{frame}

\begin{frame}{Coherence for joint and conditional events ($\C^*$)}
\begin{minipage}{\linewidth}
With coherent views on disjoint events, we seek to quantify our conditional beliefs through joint and conditional trades.
\end{minipage}
\vfill
\begin{minipage}{\linewidth}
\begin{minipage}{0.45\linewidth}
{\textbf{The conditional trade on $A_1|A_2$}}

Given the outcome
\begin{itemize}
\item[$A_2$]$\sim$ the ticket is:
\begin{itemize}
\item[$A_1$] $\sim$ redeemable for a unit of the quantity
\item[$\neg A_1$] $\sim$ worth nothing, or
\end{itemize} 
\item[$\neg A_2$]$\sim$ the trade is annulled.
\end{itemize}
\end{minipage}
\hfill
\begin{minipage}{0.45\linewidth}
\begin{minipage}{\linewidth}
{\textbf{The trade on disjoint $A_i$}}

The prices $\pr{A_i}$ are assumed to be coherent ($\C$).
\end{minipage}
\vfill
\bigskip
\begin{minipage}{\linewidth}
{\textbf{The joint trade on $A_1A_2$}}

The trades are priced as if $A_1A_2$ was a single event $\tilde{A}$.

\end{minipage}
\end{minipage}
\end{minipage}
\vfill
\begin{minipage}{\linewidth}
But we need additional rules $\C^*=\C\cup\{R_i\}$ to avoid being in a dutch book.
\end{minipage}
\end{frame}

\begin{frame}{$\C^*\impliedby\neg\D;\quad$ The space of possible outcomes of a trade}
\begin{minipage}{\linewidth}
\begin{minipage}{\linewidth}
\begingroup
\setlength{\abovedisplayskip}{3pt}
\setlength{\belowdisplayskip}{3pt}
$$W_1=\lambda_1(1-\pr{A_1A2})+\lambda_2(1-\pr{A_2})+\lambda_3(1-\pr{A_1|A_2})$$
$$W_2=-\lambda_1\pr{A_1A2}+\lambda_2(1-\pr{A_2})-\lambda_3\pr{A_1|A_2}$$
$$W_3=-\lambda_1\pr{A_1A2}-\lambda_2\pr{A_2}$$
\endgroup
\end{minipage}
\vfill

\bigskip
\begin{minipage}{\linewidth}
\begin{minipage}{0.4\linewidth}
The possible payouts are:
\begin{itemize}\itemsep0.5em
\item[$W_1$] if $A_1$ and $A_2$ happen, \\
\item[$W_2$]  if $\neg A_1$ and $A_2$ happen, \\
\item[$W_3$] if $A_2$ does not happen.
\end{itemize}
\end{minipage}
\hfill
\begin{minipage}{0.5\linewidth}
\bigskip
It is clear that we need such prices $P$, that it is impossible to uniquely determine such trades ${\lambda_j}$, that the payout $W_i$ is a loss in any outcome.
$$\exists_{\lambda_j}\forall{_i}(W_i<0)\iff\D$$
\end{minipage}
\end{minipage}
\end{minipage}
\end{frame}

\begin{frame}{$\C^*\impliedby\neg\D;\quad$ On solutions of linear equations}
\begin{minipage}{\linewidth}
\begin{minipage}{0.4\linewidth}
Three planes can intersect in
\begin{itemize}
\item[$0.$] a unique point,\\
\item[$1.$] a line,\\
\item[$2.$] a plane,\\
\item[$3.$] or do not intersect.
\end{itemize}
\end{minipage}
\hfill
\begin{minipage}{0.6\linewidth}
\begin{equation*}
M\cdot\vec{\lambda}=\vec{W}
\left\{\!
\begin{aligned}
a\lambda_1+b\lambda_2+c\lambda_3&=&W_1\\
d\lambda_1+e\lambda_2+f\lambda_3&=&W_2\\
g\lambda_1+h\lambda_2+i\lambda_3&=&W_3
\end{aligned}
\right.
\end{equation*}
\end{minipage}
\end{minipage}
\vfill
\begin{minipage}{\linewidth}
The $0.$ scenario is the only one that would allow \emph{Them} to uniquely determine the needed trades $\lambda_i$ to put \emph{Us} in a dutch book. Thus
$$\det(M)\ne0\implies\D.$$
\end{minipage}
\end{frame}

\begin{frame}{$\C^*\impliedby\neg\D;\quad$ The impossibility of a solution}
\begin{minipage}{\linewidth}
$$
\resizebox{0.95\hsize}{!}{
$
\begin{vmatrix}
1-\pr{A_1A_2}&1-\pr{A_2}&1-\pr{A_1|A_2}\\
-\pr{A_1A2}&1-\pr{A_2}&-\pr{A1|A_2}\\
-\pr{A_1}&-\pr{A_2}&0\\
\end{vmatrix}
=
\pr{A_1A_2}-\pr{A_1|A_2}\cdot\pr{A_2}
$
}
$$
\end{minipage}
\vfill
\begin{minipage}{\linewidth}
\begin{minipage}{0.47\linewidth}
If the determinant is not zero, it is always possible to determine such trades ${\lambda_j}$, that the payout $W_i$ is a loss in any outcome.
$$\det\ne0\implies\D$$
\end{minipage}
\hfill
\begin{minipage}{0.45\linewidth}
\centerline{\scalebox{0.85}{$\bm{\underbrace{\pr{A_1A_2}=\pr{A_1|A_2}\cdot\pr{A_2}}_\B}$}}

Thus
\begingroup
\setlength{\abovedisplayskip}{2pt}
\setlength{\belowdisplayskip}{2pt}
$$\det=0\impliedby\neg\D$$
$$\iff$$
$$\B\impliedby\neg\D$$
\endgroup
\end{minipage}
\end{minipage}
\end{frame}

\begin{frame}{$\C^*\implies\neg\D;\quad$ Describing the trades}
\begin{minipage}{\linewidth}
\begin{minipage}{0.45\linewidth}
\begingroup
\setlength{\abovedisplayskip}{3pt}
\setlength{\belowdisplayskip}{3pt}
The derived model in $\C$ is easily extended to cover the trade on $A_1A_2$ ; a trade on a simple event with indicator an $I_{A_1A_2}$. 

\phantom{.}
\endgroup
\end{minipage}
\hfill
\begin{minipage}{0.45\linewidth}
The conditional trade on $A_1|A_2$ costing $\pr{A_1|A_2}$ and paying $1$ if $A_1$, but only if $A_2$ is described by,
$$I_{A_2}(I_{A_1}-\pr{A_1|A_2})$$
\end{minipage}
\end{minipage}
\vfill
\begin{minipage}{\linewidth}
After allowing trades on all events, the complete payout is described by
$$W=\underbrace{\sum_{i=1}^2\lambda_i(I_{A_i}-\pr{A_i})}_{W^\prime}+\underbrace{\lambda_3(I_{A_1A_2}-\pr{A_1A_2})}_{W_{1,2}}+\underbrace{\lambda_4I_{A_2}(I_{A_1}-\pr{A_1|A_2})}_{W_{1|2}}$$
\end{minipage}
\end{frame}

\begin{frame}{$\C^*\implies\neg\D;\quad$ Implications of $E(W)$}
\begin{minipage}{\linewidth}
\begin{minipage}{0.3\linewidth}
By applying the same mechanics as in the proof of $\C$, we see that both $E(W^\prime)$ and $E(W_{1,2})$ are zero.
\end{minipage}
\hfill
\begin{minipage}{0.65\linewidth}
\begingroup
\setlength{\abovedisplayskip}{3pt}
\setlength{\belowdisplayskip}{3pt}
\begin{equation*}
\begin{split}
E(W_{1|2})&=E\Big[\lambda_4I_{A_2}(I_{A_1}-\pr{A_1|A_2})\Big]\\
&=\lambda_4E(I_{A_2}I_{A_1})-\lambda_4E(I_{A_2}\pr{A_1|A_2})\\
&=\lambda_4\Big(E(I_{A_2}I_{A_1})-\pr{A_1|A_2}E(I_{A_2})\Big)\\
&=\lambda_4\Big(\pr{A_1A_2}-\pr{A_1|A_2}\pr{A_2}\Big)
\end{split}
\end{equation*}
\endgroup
\end{minipage}
\vfill
% \bigskip
% \hfill

% \begin{minipage}{\linewidth}
% $$E(W_{1,2})=0\implies\neg\D$$
% $$\iff$$
% $$\scalebox{0.9}{$\bm{\B\sim\pr{A_1A_2}=\pr{A_1|A_2}\cdot\pr{A_2}}$}$$
% \end{minipage}
\end{minipage}
\vfill
\begin{minipage}{\linewidth}
%$$\scalebox{0.9}{$\bm{\B\sim\pr{A_1A_2}=\pr{A_1|A_2}\cdot\pr{A_2}}$}$$
If we require $\overbrace{\pr{A_1A_2}=\pr{A_1|A_2}\cdot\pr{A_2}}^\B$, we have $E(W)=0$ and can again apply the Corollary; i.e. no dutch book. Thus
$$\B\implies\neg\D.$$
\end{minipage}
\end{frame}

\begin{frame}{$\C^*\iff\neg\D;\quad$ Theorem of Conditional Coherence}
\begin{minipage}[]{0.5\linewidth}
{\textbf{Theorem} (Coherence$^*$)}
Your prices $\pr{A_i}$ and $\pr{A_i|A_j}$ at which you are willing to buy and sell tickets cannot lead you into a dutch book if and only if they are coherent,

$$\C^*\iff\neg\D.$$
\end{minipage}
\hfill
\begin{minipage}[]{0.3\linewidth}
%\centering
$$
\C^*
=
\left\{\!\begin{matrix}
\pr{A_i}\ge0, \\[1ex]
\pr{A_{C}}=1, \\[1ex]
FA,\\[1ex]
\B\\[1ex]
\end{matrix}\right\}
\quad
$$
\end{minipage}
\end{frame}

\begin{frame}{Conclusions: Uncertain and Random events}
%There was no mentioning of randomness.
\begin{minipage}{\linewidth}
\begin{minipage}{0.6\linewidth}
\begin{minipage}{\linewidth}
There was no mention of randomness during our construction. Coherence ($\C^*$) allows us to treat all uncertain ($\mathcal{U}$) events about which we hold subjective beliefs, while the frequentist interpretation of probability can only treat random events ($\mathcal{R}$).
\end{minipage}

\vfill

\bigskip

\begin{minipage}{\linewidth}
\begingroup
\setlength{\abovedisplayskip}{2pt}
\setlength{\belowdisplayskip}{2pt}
Note that the claim
$$\textbf{random}\subset\textbf{uncertain}$$
%$$\mathcal{R}\subset\mathcal{U}$$
is justified  by de Finetti's theorem.
\endgroup
\end{minipage}
\end{minipage}
\hfill
\begin{minipage}{0.35\linewidth}
\centering
\begin{minipage}{\linewidth}
\tikzset{set/.style={draw,circle,inner sep=0pt,align=center}}
\begin{tikzpicture}[scale=1,transform shape]
  (nat) at (0,-0.4)  (rea) {};
\node[set,fill=blue!20,text width=2.2cm,label={[below=60pt of int]$\mathcal{U}$}] 
  (int) at (0,-0.2)  {};
\node[set,fill=red!20,text width=1cm] (nat) at (0,0.2) {$\mathcal{R}$};

\draw[decoration={brace,mirror,raise=5pt, amplitude=10pt},decorate]
  (-1,1) -- node[left=15pt] {$\C^*$} (-1,-1.5);
\end{tikzpicture}
\end{minipage}
\hfill
\bigskip

\begin{minipage}{\linewidth}
\begin{itemize}
\item[$\mathcal{R}$] are random events\\
\item[$\mathcal{U}$] are uncertain events
\end{itemize}
\end{minipage}
\end{minipage}
\end{minipage}
\end{frame}

\end{document}
\grid
